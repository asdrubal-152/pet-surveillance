%------------------------------------------------------------------------------%
\section{Introducción}
%------------------------------------------------------------------------------%

En México, 7 de cada 10 hogares mexicanos tienen algún animal doméstico, y de estos, 89\% son perros \parencite{inegi2016}. Una gran parte de estas mascotas pasa su tiempo en solitario en sus viviendas, cuando sus dueños se marchan a realizar sus activades laborales.

El vínculo que se forma entre una mascota y su amo puede llegar a ser muy profundo. Por esta razón, algunas mascotas y dueños pueden llegar a sentir ansiedad al separse de sus mascotas. En el caso de los perros, esta ansiedad se puede manifestar en comportamientos destructivos hacia los objetos de su hogar \parencite{parthasarathy2006}. 


\subsection{Justificación}
\label{sec:justificacion}

Este trabajo presenta una solución tecnológica para dos de los problemas causados por dejar a las mascotas en solitario:

\begin{enumerate}
  \item Disminuir la ansiedad de los dueños al dejar a su mascota sin vigilancia.
  \item Reducir o impedir el comportamiento destructivo cuando la mascota está en solitario. 
\end{enumerate}

Esta problemática se origina en las exigencias de los horarios laborales, en los que una persona promeido debe dedicar hasta 40 horas a la
semana de tiempo en la oficina. La ansiedad causada por la falta de atención y la soledad es la causa de los destrozos y malos comportamientos por parte de los animales \parencite{ibanes2022}.

Una solución que ha surgido para paliar este problema es el uso de paseadores. Sin embargo, el tiempo del paseo por lo regular no es mayor a una hora, por día. Insuficiente para la enorme cantidad de tiempo que los animales pasan en soledad día a día y además esto sólo es para los perros, ignorando a todo el grupo de los felinos.


\subsection{Objetivos}

El presente producto plantea el uso de un sistema de visión artificial para monitorear la actividad de mascotas en tiempo real y reforzar ciertas pautas de entrenamiento, principalmente evitar la destrucción de mobiliario y el uso de este por parte de los animales. 

\subsection{Estructura}

A continuación se hace una breve descripción de las secciones de este trabajo.

\begin{itemize}
  \item Objetivos.
  \item Desarollo conceptual.
\end{itemize}